\chapter{Introduction}
\section{Act-R}
Adaptive Control of Thought—Rational

ACT-R is a cognitive architecture: a theory about how human cognition works. The goal of cognitive science is to understand the nature of the human mind, and the aim of ACT-R is to model the behaviour of the human brain. 
From an external point of view ACT-R can be intended as a programming language; however its internal structure is based on assumptions about human cognition. The assumptions are established by psychologists, who research and study the human cognition, through numerous experiments ~\cite{Allen94}. ACT-R's most important assumption is that human knowledge can be divided into two irreducible kinds of representations: declarative and procedural.

ACT-R architecture is the ultimate successor of a series of increasingly precise models of human cognition, developed by John R. Anderson. Anderson credits Allen Newell as source of influence to his theory.

One important feature of ACT-R that distinguishes it from other theories in the field is that it allows researchers to collect quantitative measures that can be directly compared with the quantitative measures obtained from human participants. The measures compared between humans and ACT-R are: time needed to perform the task and accuracy in the task.

As programming language, ACT-R is a framework: it provides generic features and functionality, that can be specialized for different task. In ACT-R for different tasks 
(e.g., Tower of Hanoi, memory for text or for list of words, language comprehension, 
communication, aircraft controlling), researchers create "models" (i.e., programs) in ACT-R. 
These models reflect the modelers' assumptions about the task within the ACT-R view of 
cognition.
\section{OpenCV}
\section{The objective}
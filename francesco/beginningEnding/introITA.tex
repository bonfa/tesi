\chapter*{Estrazione Automatica di Informazione Simbolica Da Immagini}
\addcontentsline{toc}{chapter}{Estrazione Automatica di Informazione Simbolica Da Immagini}
	
	\section*{Introduzione}
	\addcontentsline{toc}{section}{Introduzione}
	Questo documento descrive i principali passi dello sviluppo di un modulo software di computer vision che effettua l'analisi di immagini e comunica i risultati all'architettura cognitiva \mbox{\emph{ACT-R}\footnote{ACT-R è un'architettura cognitiva, cioè un framework che modella la struttura e il comportamento umano. Per maggiori informazioni, vedere~\url{act-r.psy.cmu.edu}.}}.
	In particolare, il programma ha il compito di riuscire a riconoscere le forme geometriche contenute nelle immagini, i loro colori e a effettuare valutazioni qualitative e quantitative su tali oggetti.

	L'attività è stata svolta presso il \emph{Centro di Scienze Cognitive} dell'\emph{Albert-Ludwigs-Universität Freiburg}.
	Uno delle tematiche principali della ricerca di 
	 
			
	\section*{L'ambiente di lavoro}
	\addcontentsline{toc}{section}{L'ambiente di lavoro}


	\section*{Lo stato dell'arte}
	\addcontentsline{toc}{section}{Lo stato dell'arte}


	\section*{Gli obiettivi}
	\addcontentsline{toc}{section}{Gli obiettivi}


	\section*{Il processo di sviluppo}
	\addcontentsline{toc}{section}{Il processo di sviluppo}


	\section*{Il design del software}
	\addcontentsline{toc}{section}{Il design del software}


	\section*{Implementazione e testing}
	\addcontentsline{toc}{section}{Implementazione e testing}
	

	\section*{Conclusioni}
	\addcontentsline{toc}{section}{Conclusioni}


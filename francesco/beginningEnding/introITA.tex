\chapter*{Estrazione Automatica di Informazione Simbolica Da Immagini}
\addcontentsline{toc}{chapter}{Estrazione Automatica di Informazione Simbolica Da Immagini}
	
	\section*{Introduzione}
	\addcontentsline{toc}{section}{Introduzione}
	Questo documento descrive i principali passi dello sviluppo di un modulo software di computer vision che effettua l'analisi di immagini e comunica i risultati all'architettura cognitiva \mbox{\emph{ACT-R}}\footnote{ACT-R è un'architettura cognitiva, cioè un framework che modella la struttura e il comportamento umano. Per maggiori informazioni, vedere~\url{act-r.psy.cmu.edu}.}.
	In particolare, il programma ha il compito di riuscire a riconoscere le forme geometriche contenute nelle immagini, i loro colori ed effettuare valutazioni qualitative e quantitative su tali oggetti.

	L'attività è stata svolta presso il \emph{Center for Cognitive Science} dell'università \emph{Albert-Ludwigs-Universität Freiburg} della città di Friburgo.
	Le attività di ricerca del centro, che come suggerisce il nome hanno come ambito principale le \emph{scienze cognitive}\footnote{\emph{Le scienze cognitive sono un gruppo di discipline che hanno come scopo lo studio delle capacità cognitive delle menti naturali o artificiali, della possibilità di trasmettere questo sapere agli altri e di averne consapevolezza. La scienza cognitiva è la specifica materia, tra le scienze cognitive, che spiega i modi in cui menti naturali o artificiali filtrano e colgono informazioni percettive, le rielaborano e riescono a intraprendere delle decisioni in base alle circostanze esperite, tanto da "reagire" al mondo esterno anche elaborando degli artefatti}~\cite{legrenzi2005prima}.}, 	
	al momento si focalizzano sul ragionamento spaziale\footnote{Il ragionamento spaziale è una disciplina che si occupa del ragionamento basato sugli oggetti nello spazio; in particolare studia le astrazioni dei concetti spaziali della conoscenza di base sulla quale si basa la prospettiva umana della realtà fisica.} e i ricercatori utilizzano ACT-R come strumento di supporto per i loro studi. 
	
	In questo contesto, il lavoro discusso in questo documento rappresenta una parte del lavoro sviluppato da un team di tre persone, il cui obiettivo finale risulta essere quello di migliorare la percezione dell'architettura cognitiva, rendendola più simile a quella umana.  
	Una delle maggiori limitazioni di \mbox{ACT-R}, infatti, è il fatto che essa lavora in un ambiente virtuale troppo semplice per rappresentare la realtà. 
	Il software sviluppato rappresenta un punto di partenza per permettere ad \mbox{ACT-R} di elaborare oggetti direttamente dal mondo reale, superando tale limite.
	 
	Dal momento che allo stato attuale l'architettura cognitiva non presenta dei moduli propri per l'elaborazione dei dati visuali, il software sviluppato è stato creato esternamente come software indipendente. 
	Ciò ha permesso di utilizzare la libreria esterna \mbox{\emph{OpenCV}} per l'elaborazione di immagini e richiede l'introduzione di un protocollo di comunicazione client-server in modo da rendere possibile la comunicazione tra \mbox{ACT-R} e il modulo di visione sviluppato.

	\section*{L'ambiente di lavoro}
	\addcontentsline{toc}{section}{L'ambiente di lavoro}
	L'attività è stata condotta presso il \emph{Center for Cognitive Science} dell'università \emph{Albert-Ludwigs-Universität Freiburg} della città di Friburgo in Germania. Tale centro ricerca nell'ambito delle \emph{scienze cognitive}, in particolare studia il \emph{ragionamento spaziale}.

	Le scienze cognitive sono delle discipline che studiano le capacità cognitive della mente, indipendentemente dalla sua natura artificiale o naturale, e spiegano le modalità con cui essa raccoglie e filtra le informazioni percettive ricevute, le rielabora e, sulla base della conoscenza ottenuta, prende delle decisioni che causano poi le reazioni dell'agente agli eventi del mondo esterno. 
	Ciò che viene studiato, in particolare, sono le modalità con cui pensiero, emozione, immaginazione, intelletto e creatività vengono formati~\cite{legrenzi2005prima}.
	La sfida per queste discipline sta nello studiare questi aspetti data l'impossibilità di osservare i processi cognitivi umani nelle diverse fasi in cui si sviluppano. 
	
	Un aspetto caratterizzante di tutte le scienze cognitive è la loro natura multidisciplinare: esse infatti accorpano informazioni provenienti da discipline eterogenee (fisiologia, neurologia, intelligenza artificiale, filosofia e psicologia) al fine di creare un modello per la mente che sia il più generale possibile~\cite{legrenzi2005prima}. 

	La cognizione spaziale è la scienza cognitiva che studia acquisizione, organizzazione, utilizzo e revisione della conoscenza riguardante gli ambienti spaziali~\cite{r8Cspace}. L'ambito di applicabilità di tale disciplina è molto vasto e comprende sia ambienti reali che modelli astratti e prevede agenti sia umani che automatici.
	Fondamentale per questa materia è lo studio del processo di ragionamento, che comprende la successione di eventi che portano alla formazione di conoscenza a partire da una serie di premesse. 
	
	Una delle grandi sfide che il centro di ricerca si pone è quella di \emph{gettare le basi per nuove teorie cognitive, riguardanti il ragionamento e la pianificazione, applicando metodi di intelligenza artificiale ed esperimenti comportamentali}.

 

	\section*{Lo stato dell'arte}
	\addcontentsline{toc}{section}{Lo stato dell'arte}
	Di seguito si descrivono i due principali strumenti utilizzati per questo progetto: la libreria di computer vision \emph{\mbox{OpenCV}} e l'architettura cognitiva \mbox{\emph{ACT-R}}.
		
			\subsection*{ACT-R}
			\addcontentsline{toc}{subsection}{ACT-R}
				\mbox{ACT-R} è un'\emph{architettura cognitiva}, cioè l'implementazione di una teoria riguardante il sistema cognitivo umano. Come tale, esso modella struttura e comportamento del cervello umano cercando di spiegare come le diverse componenti collaborano tra loro e formano la mente umana.
				
				La teoria di \mbox{ACT-R} si basa sulla \emph{teoria unificata della cognizione} (Unified theory about cognition), sviluppata da John Robert Anderson, docente presso la Carnegie Mellon University. Il concetto fondamentale di tale teoria è che la mente umana, che è separata in diversi moduli indipendenti, ognuno dotato delle proprie funzioni, nel momento di effettuare un comportamento, agisca come un sistema integrato in cui ciascun componente svolge una precisa funzione e può comunicare con gli altri grazie a specifiche connessioni~\cite{Anderson04anintegrated}.

				Un altro pilastro fondamentale su cui si basa la teoria di \mbox{ACT-R} è la distinzione tra \emph{memoria dichiarativa} e \emph{memoria procedurale}. La prima rappresenta fatti e nozioni che l'essere umano sa ed è conscio di sapere. Per richiamare questo tipo di conoscenza, l'essere umano deve effettuare un processo consapevole. La seconda, invece, si riferisce a tutte quelle abilità e capacità che l'essere umano sa ma che ha imparato in maniera implicita. Esempi di questo tipo di conoscenza sono la lettura e la scrittura~\cite{anderson1976language}. 

				Gli elementi di base dell'architettura di \mbox{ACT-R} sono \emph{chunk} e \emph{produzioni}.
				I chunk rappresentano la memoria dichiarativa e sono delle strutture dati caratterizzate da un tipo, chiamato \emph{type}, e da una lista di coppie, ognuna delle quali è costituita da un attributo, chiamato \emph{slot}, e da un valore, chiamato \emph{value}.
				Le produzioni possono essere paragonate a funzioni e definiscono la sequenza di azioni che possono essere effettuate.
				Ogni produzione presenta un insieme di precondizioni che determinano le condizioni che devono essere verificate affinchè la funzione possa essere eseguita ~\cite{actr6refman}.

				Dal punto di vista dell'architettura, \mbox{ACT-R} è organizzato in \emph{moduli}.
				Ciascun modulo è costituito da chunk e da produzioni e svolge un insieme determinato di funzioni cognitive. 
				Essi sono l'analogo dei gruppi di neuroni che si attivano nel cervello nel momento in cui viene effettuata una determinata azione da parte dell'essere umano.
				I moduli sono indipendenti ma possono comunicare tra loro tramite dei \emph{buffer}. 
				La comunicazione, che essenzialmente è uno scambio di chunk, avviene necessariamente in maniera seriale, mentre le operazioni dei diversi moduli possono avvenire in parallelo ~\cite{actr6refman}.
				
				\mbox{ACT-R} è in grado di effettuare infiniti ragionamenti, indipendentemente dal compito da eseguire. 
				Per essere in grado di eseguire una singola attività, tuttavia, necessita di uno specifico \emph{modello}.
				Ciascun modello contiene le assunzioni di chi scrive il modello riguardo al determinato compito da eseguire.
				Tali assunzioni sono espresse sottoforma di produzioni e interagiscono con i moduli durante l'esecuzione, atto che produce una serie di operazioni atomiche cognitive che passo dopo passo portano alla soluzione del compito~\cite{Sears2012}. 
				
				Le operazioni presentano delle misure qualitative e quantitative sulla qualità dell'azione stessa, come la correttezza della soluzione trovata e il tempo necessario per completare l'operazione. 
				Ciò conferisce ai modelli la possibilità di predire la sequenza delle azioni cognitive prodotte dagli esseri umani quando provano a risolvere lo stesso compito. 
				Effettuare dei confronti con le prestazioni degli esseri umani permette di misurare la qualità del modello~\cite{Sears2012}.

				Dal punto di vista informatico, \mbox{ACT-R} è scritto in lisp e fornisce una sintassi specifica, simile al lisp, per scrivere i modelli.

			\subsection*{OpenCV}
			\addcontentsline{toc}{subsection}{OpenCV}
				%intro
				\mbox{OpenCV}, acronimo di Open Computer Vision, è una libreria per la computer vision, sviluppata inizialmente da Intel e supportata poi dall'incubatore tecnologico Willow Garage.
				La libreria è multipiattaforma ed è rilasciata sotto una licenza BSD che la rende gratuita ed open source. 
				È stata sviluppata per supportare applicazioni in tempo reale e perciò è caratterizzata da una grande efficienza computazionale.
				La versione 2.4 contiene più di 2500 algoritmi che coprono molte aree della computer vision.

				% computer vision
				La computer vision è essenzialmente la trasformazione di un'immagine o di un video in una nuova rappresentazione che può essere anche di natura completamente differente rispetto al dato iniziale. 
				L'obiettivo potrebbe essere, ad esempio, una versione in scala di grigi dell'immagine originale, oppure si potrebbe desiderare che il calcolatore sia in grado di prendere una decisione a partire dalle informazioni estratte dall'immagine in ingresso.
				In entrambi i casi si sta parlando di computer vision~\cite{bradski2008learning}.
		
				In generale, la complessità di operazioni di questo tipo risulta essere molto elevata. 
				Ciò che per un essere umano viene percepito come un'attività molto semplice, infatti, può risultare un'operazione molto complessa per un calcolatore.
				Quest'ultimo rappresenta l'immagine tramite matrice di numeri. 
				Tale rappresentazione rende complesse molte delle operazioni di visione, comprese quelle più semplici come, per esempio, il riconoscimento di un semplice oggetto.
				Purtroppo al momento questa modalità di descrizione è la migliore alternativa tra le possibilità esistenti~\cite{bradski2008learning}.

				Oltre alla rappresentazione matriciale, uno dei motivi che rendono così complicata la computer vision è la necessità molto frequente di ricostruire un mondo tridimensionale a partire da un'immagine bidimensionale. 
				Questo è un problema mal posto. 
				Diretta conseguenza di ciò è l'esistenza di infinite soluzioni, cioè infinite ricostruzioni tridimensionali del mondo a partire dall'immagine bidimensionale.
				In particolare, la soluzione dipende dall'angolo con cui si analizza l'immagine~\cite{bradski2008learning}.

				Un fattore che aumenta ulteriormente la complessità dei compiti è la presenza del rumore, che raggruppa tutte le possibili cause di variabilità presenti nel mondo: tempo atmosferico, luminosità, riflessi, movimenti, deformazioni dovute alle lenti o ai sensori e altri fenomeni.
				Il rumore può essere classificato in noto a priori, quindi prevedibile, e non noto a priori, quindi non prevedibile. 
				Il primo, essendo conosciuto, può essere corretto a tal punto da essere completamente eliminato.
				È questo, ad esempio, il caso della deformazione delle lenti delle videocamere.
				Il secondo, invece, essendo casuale, non può essere del tutto eliminato ma può essere solamente ridotto.
				Tipicamente si utilizzano metodi statistici per gestire questo secondo tipo di rumore~\cite{bradski2008learning}. 
			 
				Un metodo per facilitare le operazioni di computer vision è introdurre dell'informazione contestuale. 
				In particolare, più il problema è vincolato, più è facile utilizzare i vincoli per semplificare il problema, più la soluzione ottenuta è affidabile.
				In questo modo ovviamente la soluzione perde di generalità e diventa specifica per il problema.
				Le tecniche di machine learning, che estraggono automaticamente l'informazione contestuale e la utilizzano per risolvere i problemi, possono essere sfruttate nelle operazioni di computer vision~\cite{bradski2008learning}.
				
				Storicamente, il progetto OpenCV viene avviato nel 1999 da Intel Research, divisione del gruppo Intel dedicata alla ricerca. 
				L'iniziativa è finalizzata a migliorare le applicazioni che richiedono un alto carico di lavoro della CPU, in particolare quelle legate alla visione artificiale. 
				Le intenzioni iniziali comprendono la creazione di codice ottimizzato per un'infrastruttura standard di base per la visione, la diffusione di essa tra gli sviluppatori, la portabilità e la possibilità di creare sia applicazioni di natura commerciale che gratuite. 
				Nel 2000 viene rilasciata la prima versione alpha, seguita da cinque versioni beta tra il 2001 e il 2005 che portano nel 2006 a \mbox{OpenCV} 1.0.	
				Nel 2008 l'incubatore tecnologico Willow Garage comincia a supportare il progetto e nel 2009 viene rilasciata la versione 2.0. 
				Come stabilito dal nel piano di rilascio, ogni sei mesi viene resa pubblica una nuova versione di \mbox{OpenCV}~\cite{OpenCV:ChangeLogs}.

				La libreria offre numerose funzioni che coprono diversi rami della computer vision.
				In particolare offre strutture dati per la gestione di immagini e video; funzioni per la visualizzazione e per la gestione degli eventi da parte di tastiera e mouse; procedure per l'interazione con il filesystem; possibilità di manipolare le immagini sia tramite matrici che tramite funzioni di algebra vettoriale; supporto alle strutture dati più comuni e funzioni per tutte le operazioni di base di computer vision: filtraggio, riconoscimento di contorni e di angoli, conversione dei colori, campionamento e interpolazione, operazioni morfologiche e istogrammi.
				Inoltre integra molte funzioni per l'analisi strutturale dell'immagine, la calibrazione di telecamere, analisi del movimento, riconoscimento di oggetti e machine learning~\cite{Agam2006}.
				
				Dal punto di vista informatico, la libreria è scritta in C e C++ e presenta interfacce per i linguaggi Python, Java, Ruby e Matlab.
				Dalla versione 2.4 l'architettura è organizzata in moduli, ognuno dei quali svolge specifiche funzioni~\cite{OpenCVDoc}. 
				


	\section*{Gli obiettivi}
	\addcontentsline{toc}{section}{Gli obiettivi}


	\section*{Il processo di sviluppo}
	\addcontentsline{toc}{section}{Il processo di sviluppo}


	\section*{Il design del software}
	\addcontentsline{toc}{section}{Il design del software}


	\section*{Implementazione e testing}
	\addcontentsline{toc}{section}{Implementazione e testing}
	

	\section*{Conclusioni}
	\addcontentsline{toc}{section}{Conclusioni}



%\chapter{Appendix B}\label{appB}
\lstset{breaklines=true, basicstyle=\small\ttfamily, columns=fullflexible, keepspaces=true,stepnumber=2}


	
	\section{Message Structure}\label{appB}
	During the communication between the \mbox{Lisp} interpreter and the Vision Module, two types of message are used:
	\begin{itemize}
		\item the requests, sent from the client to the server;
		\item the responses, sent from the server to the client.
	\end{itemize}
	 
	\subsection{Request Format}
		The request message consists of two parts, one fixed and one variable.
		The fixed part is the attribute \lstinline!cmd!. 
		The variable part is the value. 
		The only meaningful value for this work is \lstinline!getFeature!.
		This command tells the server to extract all the possible features from the input image.
		Listing \ref{lst:mexReq} shows the message for this request:
	
		\lstset{frame=single, breaklines=true, basicstyle=\small\ttfamily, columns=fullflexible, keepspaces=true,stepnumber=2, caption=\textit{Request Message}, label=lst:mexReq, captionpos=b}
		\begin{lstlisting}
		 {"cmd":"getFeature"}		\end{lstlisting}
	

	\subsection{Response Format}
		The response message contains a list of all the shapes recognized in the input image.
		Each item of the list is represented by a list of attributes:
		\begin{itemize}
			\item \lstinline!Bbox!: a rectangular bounding box, defined by two points;
			\item \lstinline!Color!: the color of the shape;
			\item \lstinline!Type!: the type of the shape;
			\item \lstinline!Vertices!: the vertixes of the shape, if present.
		\end{itemize}
		Listing \ref{lst:mexResp} shows an example of response, in which a rectangle and a triangle are detected:

		\lstset{frame=single, breaklines=true, basicstyle=\small\ttfamily, columns=fullflexible,  keepspaces=true,stepnumber=2, caption=\textit{Example of Response Message},label=lst:mexResp,captionpos=b}
		\begin{lstlisting}
		[
		 {
		   "Bbox":[
			     {
			        "x":30,
			        "y":7
			     },
			     {
			        "x":44,
			        "y":22
			     }
			  ],
		   "Color":"green",
		   "Type":"Quadrilateral",
		   "Vertices":[
			         {
				    "x":30,
				    "y":7
				 },
				 {
				    "x":30,
				    "y":21
   				 },
				 {
				    "x":44,
				    "y":22
				 },
				 {
				    "x":44,
				    "y":7
				 }
			      ]
		 },
		 {
		   "Bbox":[
			     {
			        "x":4,
                                "y":7
			     },
			     {
				"x":21,
				"y":21
			     }
			  ],
		   "Color":"red",
		   "Type":"Triangle",
		   "Vertices":[
				 {
				    "x":12,
				    "y":7
				 },
				 {
				    "x":4,
				    "y":21
 				 },
				 {
				    "x":21,
				    "y":21
				 }
			      ]
		 }
		]	
		\end{lstlisting}\label{lst:mexResp}
	

\section{Tools Used in The Development}\label{appC}	
	The tools used for the development are: 
	\begin{itemize}
	    	\item \textbf{Eclipse CDT} as IDE for writing C and C++ code;
	    	\item \textbf{CUTE} as testing tool;
		\item \textbf{Git} as distributed version control system;
		\item \textbf{Trac} as web-based project management and bug tracking system;
		\item \textbf{Make} as build automation tool;
		\item \textbf{Boost library} for managing all the issues related to multithreading;
		\item \textbf{OpenCV library} as computer vision operations;
		\item \textbf{Dia} as tool for creating and reading UML diagrams;
		\item \textbf{cpp2dia} as tool for creating authomatically UML diagrams from source code;
	\end{itemize}	






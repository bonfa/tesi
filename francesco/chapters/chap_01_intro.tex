\chapter{Introduction}
\section{Background}
\section{Problem description}
\section{Working Instruments}
\subsection{ACT-R}
\subsection{OpenCV}
	OpenCV, an abbreviation that stands for \emph{Open Source Computer Vision}, is a computer vision library that was originally developed by Intel and, later on, by Willow Garage.
	It is a cross-platform library, released under a BSD license, thus it is free and open source. In the beginning it was developed in C and C++ and afterwards it was expanded by the addition of interfaces for Java and Python. OpenCV is designed for computational efficiency and with a strong focus on real-time applications. The version 2.4 has more than 2500 algorithms. The library has been used in many applications as, for example, mine inspection and robotics \cite{OpenCV:MainWebPage}. The following sections contain a brief history of the library and a list of its main features.
		
	\subsubsection*{History}
	The OpenCV Project started in 1999 as an Intel Reasearch initiative aimed to improve CPU intensive applications as a part of projects including real-time ray tracing and 3D display walls. The early goals of the project were developing optimized code for basic vision infrastructure, spreading this infrastructure to developers and making it portable and available for free, using a license that let the developers create both commercial and free applications.\newline
	The first alpha version was released to the public in 2000, followed by five beta versions between 2001 and 2005, which lead to version 1.0 in 2006. In 2008, the technology incubator Willow Garage begun supporting the project and, in the same year, version 1.1  was released.
	In October 2009, OpenCV 2.0 was released. It includes many improvements, such as a better C++ interface, more programming patterns, new functions and an optimization for multi-core architectures. According to the current OpenCV release plan, a new version of the library is delivered on a six-months basis. \cite{OpenCV:ChangeLogs}.
	
	\subsubsection*{Main Features}
	OpenCV offers a wide range of possibilities. First of all, it provides an easy way to manage image and video data types. It also offers functions to load, copy, edit, convert and store images and a basic graphical user interface that lets the developers handle keyboard and mouse and display images and videos. The library lets manipulate images even with matrix and vector algebra routines. It supports the most common dynamic data structures and offers many different basic image processing functions: filtering, edge and corner detection, color conversion, sampling and interpolation, morphological operations, histograms and image pyramids. Beyond this, it integrates many functions for structural analysis of the image, camera calibration, motion analysis and object recognition. \cite{Agam2006}.

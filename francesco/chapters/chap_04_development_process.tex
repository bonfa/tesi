\chapter{Development Process}
This chapter, after having introduced the software development framework SCRUM, describes the development process adopted during the development.

	% TODO
	\section{SCRUM}
	SCRUM is a framework for the agile management of the project development. 
	As such, it does not define the technical way in which the developers must do the job but it follows the development process. 	
	The framework has many dimensions, roles, events, rules and artifacts, each of which within the framework serves a specific purpose and contributes the Scrum's success. The following sections describe one by one these components.

		\subsection{The Scrum Team}
		The \emph{Scrum Team} is composed of a \emph{Product Owner}, the \emph{Development Team}, and a \emph{Scrum Master}.
		These figures have different roles but all of them have to reach the same goal, deliver an increment part of usable product at constant time intervals (see \ref{Sprint Planning}). Scrum Teams are self-organizing and cross-functional. Self-organization gives the members the possibility to choose how best to accomplish their work, without being directed by others outside the team. Cross-functionality gives the team all the competencies needed to accomplish the work without depending on others not part of the team. The delivery of the products is iterative and incremental. This fact guarantees a constant feedback on the correctness of the job and ensures that a potentially useful version of the working product is always available.
			\subsubsection{Product Owner}
			The \emph{Product Owner} is the person who represents the stakeholders of the job. The Product Owner can represent the will of a committee but must be one person. His or her role is to define the requirements that the new versions of the product must have, give them the priorities, explain them to the team in detail and is responsible for the performance of the team. These requirements are called \emph{Backlog Items} and are included in the \emph{Product Backlog}. This document is described in more details in the \ref{Artifacts} section.

			\subsubsection{Development Team}
			The \emph{Development Team} consists of professionals who have to add the new functionalities to the product. The team are usually composed of three to nine members. One team must be self-organizing, this means that only the members can decide the step by step tasks to be accomplished in order to add the functionalities to the product. Every team is also cross-functional, i.e. is composed by people who have different skills. In this way it can be autonomous and it does not have to depend on other people outside the team to accomplish its job. More the Development Team's synergy is, more optimized its overall efficiency and effectiveness are.
 
			\subsubsection{Scrum Master}
			The \emph{Scrum Master} has the role to verify that Scrum is understood and put in place. He or she does this checking that everyone in the team follows Scrum theory, practices, and rules. 
			The Scrum Master is the enforcer of the rules and interacts with all the people inside and outside the Scrum Team in order to teach which interactions are useful and which are not.
			On one side, he helps the Product Owner finding techniques for managing the \emph{Product Backlog}, teaching him how to communicate in clear way with the Development Team and in understanding and practicing agility. On the other side, he coaches the Development Team in self-organization and cross-functionality, he protects it from unhelpful interruptions and keeps it focused on the tasks. 
			For this, often this role is referred as a servant-leader for the Scrum Team.


		\subsection{The Events}	
			\subsubsection{Sprint Planning}

		
		\subsection{Artifacts}

	
	
	% TODO 
	\section{The Adopted Development Process}
		%Cose che voglio dire:
		1) Processo di sviluppo incementale e iterativo
		2) No pair programmin nè test driven development  (dirlo ? )
		3) Sviluppo agile
		4) Come abbiamo usato noi SCRUM
			-) sprint: 2 settimane
			-) non c'era un committente ma c'era il project manager
			-) 1 team

\begin{comment}
	\section{SCRUM}

	\section{Working Instruments}
	In addition to ACT-R and OpenCV, many other tools have been used. Here follows a list of the most important ones.
		\begin{itemize}
		\item \textbf{Eclipse IDE for C/C++ Developers} as \textit{integrated development environment} \footnote{More informations at \url{www.eclipse.org}};
		\item \textbf{Cute} as \textit{unit testing framework} \footnote{More informations at \url{http://cute-test.com}};
		\item \textbf{Mylyn} as \textit{task and application lifecycle management} \footnote{More informations at \url{www.eclipse.org/mylyn}};
		\item \textbf{Trac} as \textit{bug tracking system} \footnote{More informations at \url{trac.edgewall.org}};
		\item \textbf{Git} as \textit{version control system} \footnote{More informations at \url{git-scm.com}}; 
		\item \textbf{Dia} and \textbf{cpp2dia} to create UML diagrams \footnote{More informations at \url{dia-installer.de} and at \url{cpp2dia.sourceforge.net}};
		\item \textbf{ZBar} to read QR codes \footnote{More informations at \url{zbar.sourceforge.net}}.
		%%It can be useful for drawing a large variety of diagrams, in particular UML diagrams, network maps and flowcharts. 
		\end{itemize}
\end{comment}

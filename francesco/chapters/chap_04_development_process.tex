\chapter{Development Process}
This chapter, after having introduced the software development framework SCRUM, describes the development process adopted during the development.

	% TODO
	\section{SCRUM}
	SCRUM is a framework for the agile management of the project development. 
	As such, it does not define the technical way in which the developers must do the job but it follows the development process. 	
	The framework has many dimensions, roles, events, rules and artifacts, each of which within the framework serves a specific purpose and contributes the Scrum's success. The following sections describe one by one these components.

		\subsection{The Scrum Team}
		The \emph{Scrum Team} is composed of a \emph{Product Owner}, the \emph{Development Team}, and a \emph{Scrum Master}.
		These figures have different roles but all of them have to reach the same goal, deliver an increment part of usable product at constant time intervals (see \ref{Sprint Planning}). Scrum Teams are self-organizing and cross-functional. Self-organization gives the members the possibility to choose how best to accomplish their work, without being directed by others outside the team. Cross-functionality gives the team all the competencies needed to accomplish the work without depending on others not part of the team. The delivery of the products is iterative and incremental. This fact guarantees a constant feedback on the correctness of the job and ensures that a potentially useful version of the working product is always available.
			\subsubsection{Product Owner}

			\subsubsection{Development Team}
			\subsubsection{Scrum Master}





		\subsection{The Events}	
			\subsubsection{Sprint Planning}
		
		\subsection{Scrum Artifacts}

	
	
	% TODO 
	\section{The Adopted Development Process}
		%Cose che voglio dire:
		1) Processo di sviluppo incementale e iterativo
		2) No pair programmin nè test driven development  (dirlo ? )
		3) Sviluppo agile
		4) Come abbiamo usato noi SCRUM
			-) sprint: 2 settimane
			-) non c'era un committente ma c'era il project manager
			-) 1 team

\begin{comment}
	\section{SCRUM}

	\section{Working Instruments}
	In addition to ACT-R and OpenCV, many other tools have been used. Here follows a list of the most important ones.
		\begin{itemize}
		\item \textbf{Eclipse IDE for C/C++ Developers} as \textit{integrated development environment} \footnote{More informations at \url{www.eclipse.org}};
		\item \textbf{Cute} as \textit{unit testing framework} \footnote{More informations at \url{http://cute-test.com}};
		\item \textbf{Mylyn} as \textit{task and application lifecycle management} \footnote{More informations at \url{www.eclipse.org/mylyn}};
		\item \textbf{Trac} as \textit{bug tracking system} \footnote{More informations at \url{trac.edgewall.org}};
		\item \textbf{Git} as \textit{version control system} \footnote{More informations at \url{git-scm.com}}; 
		\item \textbf{Dia} and \textbf{cpp2dia} to create UML diagrams \footnote{More informations at \url{dia-installer.de} and at \url{cpp2dia.sourceforge.net}};
		\item \textbf{ZBar} to read QR codes \footnote{More informations at \url{zbar.sourceforge.net}}.
		%%It can be useful for drawing a large variety of diagrams, in particular UML diagrams, network maps and flowcharts. 
		\end{itemize}
\end{comment}

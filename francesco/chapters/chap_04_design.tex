\chapter{Design}
  The following chapter describes the architecture of the software. 
  Software design is supported by diagrams, to better explain the contents and the structure of the project itself. The following sections, after having given a brief description of the hole architecture of the sofware, describe the class hierarchy, the \todo{terminare...}.
  
  \subsection{The Whole Software}
  The aim of the 
  \begin{figure}[h]
	  \begin{center} 
	    \includegraphics[scale=0.1]{images/ch_04/classDesign.jpg}
	  \end{center} 
	  \caption{\textit{Structure of Act-r.}}  
	  \label{fig:modulesActr}
  \end{figure}
  \todo{cambiare l immagine e aggiungere la descrizione...}.

  
  \subsection{The Class Hierarchy}
  \begin{figure}[h]
	  \begin{center} 
	    \includegraphics[scale=0.2]{images/ch_04/designClassObjectHierarchy.jpeg}
	  \end{center} 
	  \caption{\textit{Structure of Act-r.}}  
	  \label{fig:modulesActr}
  \end{figure}
  The picture above describes the hierarchy of the classes defined to contain the object detected in the images. The \textit{Object class} represents a generic object that can be found in an image. Thus, everything which is different from the background can be seen as an instance of the \textit{Object Class}. Every object must be bounded by a \textit{bounding box}. This structure is represented by the \textit{BoundingBox Class}. All the bounding boxes have rectangular shape. The requirements were such that it was not necessary to define more complicate shapes \todo{migliorare questa frase, fa schifo}. The \textit{attended} attribute of the \textit{Object class} is set to true if the object has been already returned to ACT-R, otherwise it is set to false. The \textit{rotation} is a value that gives the amount of the rotation in the counterclockwise directions starting from the orizontal direction \todo{controllare la correttezza}. \todo{vedere se il metodo getChunk esiste ancora}\todo{se non esiste eliminarlo dall immagine}.
  Besides the generic objects, for sure there are two categories of object that can be found in the processed images, the \textit{QR codes} and the \textit{simples shapes}. A simple shape can be, for example, a circle, a square, a rectangle or a triangle. The QR codes is represented in the hierarchy with the \textit{QRObject}, the generic shape with the \textit{Blob class}.
  This last class, has the \textit{area} parameter, which stands for the area of the object in pixel, and the method \textit{getArea}, which returns this value. The \textit{Blob} class is realized by three classes, \textit{Circle}, \textit{Triangle} and \textit{Quadrilateral}. 
  The \textit{Circle} class has, as parameters, the \textit{center} and the \textit{radius} of the circle itself.
  The \textit{Triangle} class, instead \todo{... finire}

  
  
  
  
  
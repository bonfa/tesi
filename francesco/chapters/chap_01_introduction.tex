\chapter{Introduction}
	% objective of the thesis
	This document describes the development of a software module which implements simple visual operations and communicates their results to the cognitive architecture \mbox{ACT-R}.
	In particular, the software has to analyze images and process them by recognizing the objects they contain, their colors, their dimensions and their positions.
	The ultimate objective of such module is to make \mbox{ACT-R} perception more similar to human one.
	
	% motivations of the choice
	One of the main limitations of the cognitive architecture is that it works in a virtual environment, too easy for representing the reality. 
	The new software represents a starting point for allowing \mbox{ACT-R} to process objects directly in the real world, overcoming this limit.
	
	% How to face the problem
	Stated the impossibility of processing the visual data directly inside the cognitive architecture itself, the module has been created externally as an independent software.
	This fact brings two main consequences: the use of OpenCV library for implementing the visual operations and the introduction of a client server architecture in order to implement the communication between \mbox{ACT-R} and the software.

	% Topic of each chapter
	Chapter \ref{ch:state_of_the_art} presents the main instruments that compose the software, \mbox{ACT-R}, the cognitive architecture and \mbox{OpenCV}, the computer vision library.
	Chapter \ref{chObjective} explains in detail the objective of the work and the requirements.
	Chapter \ref{devProcChap} describes the Scrum method, which is the adopted development method, and justifies why such framework has been chosen.
	Chapter \ref{chDesign} presents the most important aspects of the design of the software.
	Chapter \ref{impl_test} discusses some aspects of testing and implementation, 


	% unifreiburg
 	
	

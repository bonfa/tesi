\chapter{Introduction}
	% objective of the thesis
	This document describes the development of a software module which implements simple visual operations and communicates their results to the cognitive architecture \mbox{ACT-R}.
	In particular, the software has to analyze images and process them by recognizing the objects they contain, their colors, their dimensions and their positions.
	

	% unifreiburg
	The activity has been carried on in the \emph{Center for Cognitive Science} of \emph{Albert-Ludwigs-Universität Freiburg}. 
	%Currently, the actual field of research of the Center is spatial reasoning. 
	As the name of the center suggests, cognitive science is the field of research of the center.
	The studies, currently, are focused on the spatial reasoning and the researchers use \mbox{ACT-R} as a support instrument for them. 
	%The research field of the center is the cognitive science and current  spatial reasoning
	In this context, the work discussed in this document represents a part of the work developed by a team of three people, whose ultimate objective is to make \mbox{ACT-R} perception more similar to human one.

	% motivations of the choice
	One of the main limitations of the cognitive architecture is that it works in a virtual environment, too easy for representing the reality. 
	The new software represents a starting point for allowing \mbox{ACT-R} to process objects directly in the real world, overcoming this limit.
	
	% How to face the problem
	Stated the impossibility of processing the visual data directly inside the cognitive architecture, the module has been created externally as an independent software.
	This fact brings two main consequences: the use of OpenCV library for implementing the visual operations and the introduction of a client server architecture in order to implement the communication between \mbox{ACT-R} and the software.

	
	% Topic of each chapter
	This document is organized in chapters, each of which focuses on specific aspects of the creation process.
	Chapter \ref{ch:state_of_the_art} presents the context of work by describing the cognitive architecture \mbox{ACT-R} and the computer vision library \mbox{OpenCV}.
	%The following ones describe the software using the steps of the development process like a guideline. 
	Chapter \ref{chObjective} explains in detail the objective of the work and defines the requirements.
	Chapter \ref{devProcChap} describes Scrum, which is the chosen development framework, and motivates its adoption.%why the team chose such framework.
	Chapter \ref{chDesign} presents the main aspects of the architecture of the software and chapter \ref{impl_test} deepens the description of computer vision algorithms and describes the choices about the testing operations.
	After the Conclusions, in Appendix \ref{app} the reader can find other implementation details about the software.

	
	
	
	
	
	

 	
	

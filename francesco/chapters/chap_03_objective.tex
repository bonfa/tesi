\chapter{Objective}

The objective of the work is composed by the goals the software have to reach and the definition of the requirements for achieving them. These two aspects are described in the following sections.

	\section{The Goals of the Software}
\begin{comment}to be developed\end{comment}
	The goals of the software  are to simplify the psychologists work of defining new test cases for experiments in ACT-R and to recognize objects during the navigation of robots.

	At the moment, a psychologist, in order to insert a new test in ACT-R, has to create two interfaces, one for the user and one for ACT-R.
	The interface for the user is generally an image and can be created with any graphical editor. 
	The interface for ACT-R, instead, is much more complicated and requires the creator of the experiment to write each shape directly on the visual buffer. \todo{check correctness of this} \todo{check if the visual buffer has been introduced}
	
	The following picture shows an example of the two images used in the test.
	The one on the left is the image shown to the user while the image on the right represents the ACT-R one. 
	\todo{insert the photo}

	%here i would insert the photo

	So, to create a test, the psychologist has to create the same image in two different formats.
	This double work causes a loss of time for the psychologist. Moreover, writing by hand all the shapes on ACT-R visual buffer is a low-level activity and can lead to a stressing mood and consequently to errors.

\begin{comment}to be developed\end{comment}
	The purpose of the software is to avoid to the psychologists the activity of writing on ACT-R visual buffer. To achieve this, it analyzes the image created for the user, processes it extracting the features needed by ACT-R and communicates them to the artificial intelligence framework, which designs them automatically on its visual buffer. 
	This relieves the creators of the test of the double task described above leading to a better working experience, a less stressed work and a lower probability of errors.
	

	The other scope in which the software is used is the robot navigation. 
	The shapes identified by the software are used like a starting point for an object recognition process. 
	The information about these objects, then, is used by ACT-R in order to take intelligent decisions during the robot navigation inside the building, without any other knowledge of the environment. The object recognition module has not yet been developed, so the requirements for achieving this second goal are not defined. \todo{will i put this? or will i put it in the future developments? at the beginning it was a requirement. Will i say that the object recognition part is not developed yet?}

\begin{comment}
	The feature recognition, in particular, will be used by ACT-R in order to recognize objects starting from the simple shapes recognized by the software
 further use of the software is that the feature extraction process that is going to be implemented is used like an object recognition to be used during the robot navigation inside a building. 
\end{comment}	

	\section{Requirements}
	The requirements are grouped in the following sections according to the functional/non-functional classification. 
	%The simple shapes recognized are circles, triangles, squares, rectangles and ellipses.
	%Another important feature that must be introduced is the recognition of the text

		\subsection{Functional Requirements}
		The objective of the work is to design and implement a standalone software module which receives as input an image and is able to analyze it and extract the most important features of it.

		The input image is a color image which contains simple shapes. The shapes must not be overlapped and must have the same color hue.

		The software must recognize shapes and for each of them must calculate area, perimeter, colors, dimensions, rotation, a bounding box and a center. It also has to be able to recognize the color of a single pixel, calculate distances between objects, make dimensional comparisons between objects and calculate the relative position of one object in respect with another one.

		The shapes to be recognized are circles, triangles, rectangles and squares. 
		The color of the object is calculated averaging all the pixels of the object.
		The rotation is defined as the angle between the less sloped segment of the boundary of the object and the horizontal direction, calculated in the counterclockwise order.
		The bounding box is a rectangle whose coordinates are calculated getting the minimum and the maximum coordinates in the horizontal and vertical directions of the pixel of the image. 
		The center of a shape is the center of the bounding box.
		The distances between objects are calculated in two different ways. The first one is the distance between the centers of two shapes. The second one is the minimum distance calculated between each vertex of the bounding box of the first shape and all the vertexes of the bounding box of the second shape.
		The dimensional relation between two objects is made comparing their areas. 
		The relative position of two objects is made comparing the positions of the two top-left vertexes of each bounding box.
		\todo{make a bulleted list?}

		The software must be able to communicate with ACT-R. 
		In particular, ACT-R must signal to it which is the image to process and ask for a subset of the features that can be extracted. The module must return a chunk containing the information requested by ACT-R.

		\subsection{Non-Functional Requirements}
		The non-functional requirements described in the following subsections are respectively the requirements on the product and the organizational ones.

			\subsubsection{Product Requirements}
			The software module must be general purpose, portable and must work in background.
			As general purpose is intended that, if possible, it should be easily adapted to be able to work with all the experiments that will be put in place by the psychologist in the future. Moreover it must be possible to adapt the software in order to use it as a shape recognition tool in the navigation with robot. %TODO reference più avanti
			It must work on Windows, Linux and Mac OS operative systems.
	
			\subsubsection{Organizational Requirements}
			The language of the implementation of the software must be C++. 
			The computer vision library to be used must be OpenCV.
			The adopted version control system must be Git.
			A strict monitoring of the work is required.





\begin{comment}
	\section{Functional Requirements}
		The objective of this work is to make easier the creation of an ACT-R test.\newline
		At the moment, a psychologist, in order to create a test, has to create two interfaces, one for the user and one for ACT-R. The user interface is quite simple, it is formed by simple shapes that can have different colours and words. The user has to read or watch the screen and then choose between a series of possibilities. In order to be analysed by ACT-R, this interface must have a corresponding one, which is simpler and contains only the most important features of the first one. This "simpler" interface is given as input to ACT-R and is written according to a fixed pattern. The figure below shows an example of the user interface of a test and of its corresponding one.\newline \newline	

			%qui metterei le due figure


		The purpose of this work is to create a module which is able to receive as input the user interface, recognize the main features in it and create the corresponding ACT-R interface, which must contain only such features. In order to do this, the software must be made by two parts. The first part will analyze the image. It will recognize and distinguish simple shapes, such as squares, rectangles, circles, stars, arrows; distinguish the colours of the objects; determine the position of each object in the image and recognize the text in the image. The second part will create the correspondent interface for ACT-R, thus... \newline \newline	 %% come deve fare a creare lquestai nterfaccia (lisp o qualcos altro)???? 
		  
		The software is going to be tested with several ACT-R test.
	
	\section{Non Functional Requirements}
\end{comment}	


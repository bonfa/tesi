\chapter{Objective}

	\section{How psycologists define the experiments with ACT-R}

	\section{The requirements}
\begin{comment}
	\section{Functional Requirements}
		The objective of this work is to make easier the creation of an ACT-R test.\newline
		At the moment, a psychologist, in order to create a test, has to create two interfaces, one for the user and one for ACT-R. The user interface is quite simple, it is formed by simple shapes that can have different colours and words. The user has to read or watch the screen and then choose between a series of possibilities. In order to be analysed by ACT-R, this interface must have a corresponding one, which is simpler and contains only the most important features of the first one. This "simpler" interface is given as input to ACT-R and is written according to a fixed pattern. The figure below shows an example of the user interface of a test and of its corresponding one.\newline \newline	

			%qui metterei le due figure


		The purpose of this work is to create a module which is able to receive as input the user interface, recognize the main features in it and create the corresponding ACT-R interface, which must contain only such features. In order to do this, the software must be made by two parts. The first part will analyze the image. It will recognize and distinguish simple shapes, such as squares, rectangles, circles, stars, arrows; distinguish the colours of the objects; determine the position of each object in the image and recognize the text in the image. The second part will create the correspondent interface for ACT-R, thus... \newline \newline	 %% come deve fare a creare lquestai nterfaccia (lisp o qualcos altro)???? 
		  
		The software is going to be tested with several ACT-R test.
	
	\section{Non Functional Requirements}
\end{comment}	


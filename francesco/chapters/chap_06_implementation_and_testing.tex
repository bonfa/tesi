\chapter{Implementation And Testing}\label{impl_test}
	
	\section{Implementation}
	
		\subsection{Rectangles and Triangles Detection Algorithm}
		The algorithm used in order to detect triangles and quadrilaterals in the images is based on a procedure of corner detection. 
		In order to capture all the shapes, such procedure is executed three times for each color level of the image, with different thresholds.

		As the procedure introduces some \emph{false corners}, the algorithm needs functions for detecting and removing them. 
		In addition to this, this technique can detect many times the same shape; thus  the algorithm requires procedures that remove the different copies of the same shape.
		
		Listing \todo{} contains the pseudo code of the algorithm.
		The first operation is an algorithm for the suppression of the noise.
		Then the corner detection procedure is realized with three different thresholds for each level of the colored image, red, green and blue.
		This guarantees that all the shapes in the image are recognized.
		
		After the corner detection algorithm, a selection based on the number of the detected vertexes is applied in order to select only the quadrilateral and the triangles.
		In order to recognize a n-sides shape, the sets containing less than n vertexes are discarded. The remaining sets are added to a list for further processing.
		For the recognition of quadrilaterals, for example, all the sets of points with more than 3 points are added to that list.
		
		The fact that not only the sets with n sides are included in the list is a consequence of the corner detection algorithm. 
		If often happens, in fact, that a straight line is divided in two or more lines with very similar slope. Every couple of lines defines a corner which is detected. As consequence of this, even a set of points containing more than n points can define a n-vertex shapes. 
		
		In order to delete these \emph{false points}, an algorithm analyzes every set of corners removing all the points that are on the straight lines obtained by each couple of vertexes, except the points that generate the straight line. 
		
		After this, another selection based on the number of vertexes is applied. This time all the the sets that not contain n vertexes are removed from the list.
		
		 
		%This technique has some weakness. For example, due to the different values of the threshold it can happen that a straight line is 
				
		
		Another problem of the algorithm is that, due to the different thresholds applied to three different levels, most of the shapes are recognized more than once.
		In particular, the corners that limit the same shape detected with different parameters can be significantly different.  
		The gap in some cases can be even greater than 15 pixels. 
		For example, the same triangle can be specified by two different sets, like {(0;0),(200;0),(100;200)} and {(2;2),(199;0),(101;197)}. 
		In order to detect and remove these copies, a function  has been introduced that checks all the possible couples of sets. 
		It controls the similarity of the two sets by analyzing the positions of the pixels, comparing the two areas, \todo{end}.
		
		Once removed the copies, the algorithm returns the list containing the remaining sets of points.

\begin{comment}		
		various parameters like the positions of pixels, the areas and the positions of each couple of 
	
		Then, the corners of each shape are "uniti" thanks to an algorithm of fi
		
		In order to detect the douls
\end{comment}
	

	

	\section{Testing}


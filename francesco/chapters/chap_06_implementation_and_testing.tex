\chapter{Implementation And Testing}\label{impl_test}
	This chapter describes the solutions implemented for recognizing shapes, and presents how the testing operation have been conducted.
	
	
	\section{Implementation}
		As regarding the detection of shapes in images, two different algorithms are implemented, one for circles and one for shapes of n sides.
		In particular, the solution for circles makes use mainly of a function provided by the \mbox{OpenCV} library therefore is not described here.
		The algorithm for detecting n-sides shapes, instead, is more complicated and detailed. 
		Below the reader can find the description of such method.
	
		\subsection{N-sides Shape Detection Algorithm}
		The algorithm used in order to detect n-sides shapes in the images is based on a procedure of corner detection. 
		In order to capture all such objects, the procedure is executed three times for each color level of the image, every time with different parameters.
		As the corner detection algorithm identifies \emph{false corners}, it requires functions for detecting and removing them. 
		In addition, the repetition of the procedure leads to the detection of many sets which describe the same shape.	
		Consequently the algorithm requires procedures that remove the different copies of the same object.


		
		Listing \ref{lst:pseudoCodeShapeExtraction} contains the pseudo code of the algorithm.
		The first operation is a technique for reducing the noise in the image, followed by the corner detection procedure.
		The latter is applied with three different thresholds for each level of the colored image, red, green and blue.
		This guarantees that all the shapes in the image are recognized.
		Both of these functions are offered by \mbox{OpenCV} library.

		
		After the corner detection procedure, an operation of selection is applied, based on the number of detected vertexes.
		In order to recognize a n-sides shape, only the sets containing more than n-1 vertexes are kept, the other are discarded.
		For the recognition of quadrilaterals, for example, all the sets with less than four elements are discarded.
		


		The fact that not only the sets with n sides are included in the list is a consequence of the corner detection algorithm. 
		It often happens, in fact, that, due to noise in the image or imperfection of the corner detection procedure, a single segment is detected as a set of two or more shorter segments. 
		Each couple of such elements defines a \emph{false corner}. 
		For this reason, even a set of points containing more than n points can define a n-vertex shapes. 


		In order to delete the false corners, an algorithm analyzes every point set and calculates the slopes of the straight lines obtained by each couple of adjacent vertexes. 
		If the slopes of two consecutive segments are very close one to each other, the vertex common to both the segments is a false corner and therefore is removed.
		

		
		After this, another selection based on the number of vertexes is applied. This time all the sets not containing n vertexes are discarded.

		\lstset{frame=single, breaklines=true, basicstyle=\footnotesize, columns=fullflexible,  keepspaces=true,stepnumber=2, caption=Pseudo code of the n-sides shape detection algorithm, label=lst:pseudoCodeShapeExtraction, captionpos=b, morekeywords={function,foreach,of,do,if,return}}
	    	\begin{lstlisting}constant thresholds[3];	

		function detect_polygon(image, sides_number){

			noise_reduction(image);
		
			image_levels[3] = split_in_level(image);
		
			corners = [];
	
			foreach level of image_levels do{

				foreach threshold of thresholds do{
					detected_corners[] = corner_detection(level,threshold);
					corners.add(detected_corners);
				}
			}	
		
			foreach corner_set of corners do{
				if (corner_set.size() < sides_number)
					corners.remove(corner_set);
			}			

			foreach corner_set of corners do{
				remove_in_line_points(corner_set);
			}

			foreach corner_set of corners do{
				if (corner_set.size() != sides_number)
					corners.remove(corner_set);
			}

			remove_duplicates(corners);	
		
			return corners;
			}
			\end{lstlisting}
		

		
		


		
		
		
		
		Another problem of the algorithm is that, due to different thresholds applied to different levels of the image, a shape is often defined by more sets of points.
		In particular, two points representing the same vertex of the same shape belonging to different sets, most of the times do not have the same coordinates.
		The gap in some cases can be even greater than 15 pixels. 
		For example, the same triangle can be specified by \{(0;0),(200;0),(100;200)\} and \{(2;2),(199;0),(101;197)\}. 
		
		

		In order to detect and remove these copies, a function has been introduced that checks all the possible couples of sets. 
		It controls the similarity of two elements by analyzing the positions of the pixels and making comparisons about the areas and the perimeter of both the shapes. 
		When two sets specify the same object, the second one is discarded. 
		
		Once removed the copies, the algorithm returns a list containing the remaining sets of points.

		At the moment this algorithm has been implemented and tested only for quadrilaterals and triangles. 
		Anyway, it can be adapted in order to recognize shapes with more than four sides.
		In fact, for each new category of shape, only two additional functions are required: a filter for removing copies and a constructor that creates the new shape. 
		To maintain coherence with other shapes, an object in the class hierarchy that implements such constructor and the other methods is required.  
		


	\section{Testing}
	The testing activity depends strictly on the class creation.
	In particular, a class creation is always followed by the creation of its correspondent test module, which contains all the procedures required for checking the correctness of constructors, destructors and methods of the class to be tested.
	
	Differently from the \emph{test driven development} approach, in this case the test cases are not written before the definition of single methods; rather they are developed when the creation of the whole class is completed.
	As consequence of this, a test case does not necessarily fail the first time it is run. 
	
	Every method is tested ensuring that all the possible flows of the execution are followed.
	According to the \emph{equivalent partitioning} technique, input data are grouped in different categories by following the taxonomy valid/not-valid and a test case with as input a representative value for each partition is created. 
	%According to the \emph{equivalent partitioning} technique, for each function a test case with a valid input data for each category has been created as well as one for each group of invalid data.
	Differently from the specification of this technique, more than one value per category is used for the test cases instead of one.

	Moreover, conforming to the \emph{boundary value testing} technique, input values on the boundaries of data categories are introduced in the test cases in order to make the test cover as wide as possible. 

	The test cases are organized in test suites. All the suites specific of a class are included in the same test module. 
	All the test operations are automatically realized thanks to the CUTE software.

	The objective of the testing operation are to evaluate correctness and reliability of the system.



